\chapter{Related work}
\label{related}

Many libraries for accessing MediaWiki written in various languages already exist.
Some relevant examples are included below.

\begin{itemize}
\item wikitools \cite{wikitools}

wikitools is a library written in Python. It uses string-based dictionaries for parameters and results.
For example, the \texttt{blocks} sample query used in Chapter~\ref{goal} would look like this:

\begin{lstlisting}[language=python]
params = {
    'action':'query',
    'list':'blocks',
    'bkip':'8.8.8.8',
    'bkdir':'older',
    'bkprop':'byid'
}
request = api.APIRequest(site, params)
request.query()
\end{lstlisting}

This is basically identical to the string-based approach mentioned in Section~\ref{alternatives},
so it also shares all of its disadvantages.

\item WikiFunctions \cite{wikifunctions}

WikiFunctions is a .NET library used primarily in a semi-automated MediaWiki editor AutoWikiBrowser.
It uses the manual approach for a small subset of available modules.
This means the user does not have to know much about the API,
but on the other hand, the user cannot use all of the functionality of the API.

\item Linq to Wikipedia \cite{linq-to-wikipedia}

Linq to Wikipedia is a .NET library that contains a simple \lstinline{IQueryable} provider for two modules
(\texttt{search} and \texttt{opensearch}) from the MediaWiki API.
A query using this library looks like this:

\begin{lstlisting}
from wikipedia in datacontext.OpenSearch
where wikipedia.Keyword == "linq"
select wikipedia
\end{lstlisting}

This query provider is very limited in that it supports
only \lstinline{Where()}, \lstinline{Take()} and \lstinline{Skip()} methods
(the \lstinline{select} clause in the above code is actually not compiled into a call to the \lstinline{Select()} method).
Other LINQ methods are attempted to be ignored and often cause exceptions at runtime.

Also, the \lstinline{where} clause has to be in a very specific shape:
it can use only the \lstinline{Keyword} property, even though the query object has other properties.
On the other hand, objects in the resulting collection have the \lstinline{Keyword} property,
even though it is always \lstinline{null}.
This is because with \lstinline{IQueryable}, the type of the result is the same as the type used in \lstinline{where}.

\end{itemize}