\chapwithtoc{Conclusion}
\label{conclusion}

The goal of this work was to implement a C\# library to access the MediaWiki \ac{API}
in a way that is readable, discoverable, strongly-typed and flexible.

These goals were successfully accomplished using a custom \acs{LINQ} provider and code generation with Roslyn \ac{CTP}:

\begin{itemize}
\item Code written using LinqToWiki is readable:
non-query modules are accessed using simple methods,
with parameters usually given as named parameters of the method.

Query modules use \acs{LINQ} methods or \acs{LINQ} query expressions,
which should be readable to any C\# programmer.
Although the fact that even parameters that do not filter the results go into \lstinline{Where()} can be confusing.

The syntax for \texttt{prop} modules, with \acs{LINQ} queries inside \lstinline{Select()},
usually following another \acs{LINQ} query, can be quite complicated,
but we believe it describes the meaning of the code quite well,
so it should still be fairly readable.

\item The various actions possible through LinqToWiki are highly discoverable:
every action begins in a single point – the \lstinline{Wiki} class.
An \ac{IDE}, such as Microsoft Visual Studio,
can then show the user the available actions, along with their description,
through autocompletion (called IntelliSense in Visual Studio).

\item The whole library is strongly-typed:
accessing modules, setting their parameters and then accessing their results
never involves using string constants,
that would represent some module, parameter or result property.
This is because actual types with methods, method parameters and properties are used.

Thanks to this, the chance for user error when using this library is greatly lowered.

\item The library is flexible: If some module in the \ac{API} of a wiki changes
(which happens regularly), the library should be still usable.
This is achieved by regenerating code for the wiki.

The same principle also applies to different wikis.
If the modules in several wikis differ, code can be generated for each wiki separately.
\end{itemize}

\medskip

Probably the biggest difference between the original goal and the final library is naming.
Names of types, methods, method parameters and properties should follow the .NET naming guidelines,
but this was not done, because we could not separate names from the \ac{API} into words.

\section*{Sample queries}

Following are samples of queries of different modules using LinqToWiki.
They show how some of the goals of this work have been achieved.

\begin{lstlisting}
string diff = wiki.compare(fromrev: 486474789,
                              torev: 487063697)
                    .value;
\end{lstlisting}

This query compares the text of two revisions and returns differences between them.
It shows that queries for modules that return a single item take their parameters as named method parameters
and directly return their result.

\begin{lstlisting}
var pages = wiki.Query.querypage(querypagepage.Uncategorizedpages)
                  .ToList();
\end{lstlisting}

This query returns a list of uncategorized pages,
as provided by the special page \texttt{Special:UncategorizedPages}.
It shows that required parameters to list-returning modules are given as method parameters
and that \lstinline{ToList()} (or, alternatively, \lstinline{ToEnumerable()})
has to be called on the result before it can be used.

\begin{lstlisting}
var pages = (from cm in wiki.Query.categorymembers()
	       where cm.title == "Category:Query languages"
	           && cm.type == categorymemberstype.subcat
	       select cm).Pages;
\end{lstlisting}

This query returns subcategories of the category \texttt{Query languages}.
The result is not directly usable, but it can be used in \texttt{prop} queries.
The query shows how LINQ can be used for querying list-returning modules
and that the \lstinline{Pages} property is used to create source for further queries.

\begin{lstlisting}
var result = pages.Select(
	p =>
	new
	{
		Info = p.info,
		Images = p.images().ToEnumerable()
	})
	.ToEnumerable();
\end{lstlisting}

This query returns information about images present on a list of pages specified by \lstinline{pages}
(see previous query).
It shows how the result of one query can be used as a source for a \texttt{prop} query,
how the LINQ method \lstinline{Select()} is used in such queries,
how anonymous types can be used to return the required information
and how \lstinline{ToEnumerable()} (or \lstinline{ToList()}) has to be also used inside \texttt{prop} queries.