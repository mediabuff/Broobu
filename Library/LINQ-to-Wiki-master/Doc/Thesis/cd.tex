\chapter{Contents of the enclosed CD}
\label{cd}

The main part of the enclosed CD is the directory \texttt{LinqToWiki},
which contains all of the source code of the LinqToWiki library.
The same code is also available from the git repository of the library:
\url{https://github.com/svick/LINQ-to-Wiki/}.

The CD also contains this document (\texttt{thesis.pdf})
and a short presentation about the library (\texttt{presentation.pdf}).

\sectionnobreak{Using the library}

As described in Chapter~\ref{ltw}, using LinqToWiki consists of several steps.
Here, they are described in detail:

\begin{enumerate}
\item Compile the \texttt{lintowiki-codegen} application.

To do this, you need Microsoft Visual Studio 2012 with the September 2012 \ac{CTP} of Roslyn installed.
In it, open \texttt{LinqToWiki.sln} and build the project LinqToWiki.Codegen.App.

This step can be skipped, compiled version of \texttt{lintowiki-codegen} is included
in the \texttt{Tools} directory on the enclosed CD.

\item Run \texttt{lintowiki-codegen} to generate wiki-specific \ac{DLL}.

Execute the application \texttt{lintowiki-codegen.exe} with a parameter specifying which wiki
to use and optionally also other parameters altering the output.
The application requires .Net 4.5.
For further information, see Section~\ref{ltw-ca}.

This step can also be skipped, the enclosed CD contains \ac{DLL} generated for the English Wikipedia
in the directory \texttt{Lib\string\LinqToWiki.Generated}.

\item Use the generated \ac{DLL} in your application.

The LinqToWiki.Samples application (more details in Section~\ref{ltw-s}) is an example of application
that uses the generated \ac{DLL} to perform queries.

To build, execute and possibly modify it, open the solution file \texttt{LinqToWiki no codegen.sln}
in Microsoft Visual Studio 2010 or 2012.
\end{enumerate}