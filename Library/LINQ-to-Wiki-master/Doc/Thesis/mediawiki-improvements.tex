\chapter{MediaWiki improvements}
\label{mw improvements}

As mentioned in Section~\ref{paraminfo}, to generate types for each module of the \ac{API},
it is necessary to know the properties contained in the module response
and how do they map to the values of the \texttt{prop} parameter.

This information was not available, in MediaWiki previously.
For this reason, we extended the \texttt{paraminfo} module
to be able to provide information about result properties
of the \ac{API} modules, using the same type system already used to describe parameters.
Also, most of the \ac{API} modules were changed so that they provide this information to the \texttt{paraminfo} module.

Specifically, this meant adding \texttt{getResultProperties()} function to the \texttt{Api\-Base} class,
which is a base class for all module classes, and then overriding it in each module's class.
This function is then called from the \texttt{ApiParamInfo} class,
so that the provided information shows in the output of the \texttt{paraminfo} module.
For an example implementation of \texttt{getResultProperties()}, see Figure~\ref{paraminfo code}.

Of the 73~modules present in the MediaWiki core (that is, without any extensions),
5 are not suitable for having their result properties described,
because their result looks different than the result of other modules (for example, there are modules that produce \acs{RSS} feeds).
Further 5 modules do use the same response format as the other modules,
but their response cannot be described in the type system used.
There are also 17 modules that can be partially represented using this type system, but not completely.

The patch that adds this ability to the \texttt{paraminfo} module and the necessary
information to most other modules was reviewed by MediaWiki developers and merged into the official repository
on 12 June 2012.
On 2 July 2012, MediaWiki version 1.20wmf6, which includes changes from this patch, was deployed to all Wikimedia sites, including Wikipedias.

An example of the added result information to the \texttt{paraminfo} response (here for the \texttt{categorymembers} module) is in Figure~\ref{paraminfo props}.

\begin{figure}[htbp]
\begin{lstlisting}[language=php]
public function getResultProperties() {
  return array(
    'ids' => array(
      'pageid' => 'integer'
    ),
    'title' => array(
      'ns' => 'namespace',
      'title' => 'string'
    ),
    'sortkey' => array(
      'sortkey' => 'string'
    ),
    'sortkeyprefix' => array(
      'sortkeyprefix' => 'string'
    ),
    'type' => array(
      'type' => array(
        ApiBase::PROP_TYPE => array(
          'page',
          'subcat',
          'file'
        )
      )
    ),
    'timestamp' => array(
      'timestamp' => 'timestamp'
    )
  );
}
\end{lstlisting}

\caption{Implementation of \texttt{getResultProperties()} \\
in the \texttt{Api\-Query\-Category\-Members} class}
\label{paraminfo code}
\end{figure}

\begin{figure}[p]

\begin{lstlisting}[language=]
<props>
  <prop name="ids">
    <properties>
      <property name="pageid" type="integer" />
    </properties>
  </prop>
  <prop name="title">
    <properties>
      <property name="ns" type="namespace" />
      <property name="title" type="string" />
    </properties>
  </prop>
  <prop name="sortkey">
    <properties>
      <property name="sortkey" type="string" />
    </properties>
  </prop>
  <prop name="sortkeyprefix">
    <properties>
      <property name="sortkeyprefix" type="string" />
    </properties>
  </prop>
  <prop name="type">
    <properties>
      <property name="type">
      <type>
        <t>page</t>
        <t>subcat</t>
        <t>file</t>
      </type>
      </property>
    </properties>
  </prop>
  <prop name="timestamp">
    <properties>
      <property name="timestamp" type="timestamp" />
    </properties>
  </prop>
</props>
\end{lstlisting}

\caption{Result properties information for the \texttt{categorymembers} module}
\label{paraminfo props}
\end{figure}

During this work, we also noticed several bugs and inconsistencies in the \ac{API}.
Because of this, we reported eight bugs to the WikiMedia bug-tracking system.
Three of them turned out to be duplicates of already reported bugs and,
as of December 2012, three of them are still waiting to be fixed.

We also submitted eight additional patches to the MediaWiki code review system.
Although only three of them actually fix behavior of the MediaWiki \ac{API},
the rest are only fixes in documentation and other mostly insignificant changes.
Of those three patches, one is still waiting for review, because it is a breaking change,
and most likely will not be accepted.